\documentclass[a4paper]{report}

\usepackage[english]{babel}
\usepackage{amsmath}

\author{Maarten de Jonge \\
        Inge Becht}
\date{\today}
\title{Various Models and Simulations}

\begin{document}
\maketitle

\chapter{Simulating a forest fire with a cellular automaton}
\label{cha:ff}

\section{Implementation} 
\label{sec:ff_impl}

The simulation is implemented as a cellular automaton using the following simple
rules:
\begin{enumerate}
    \item A cell can be empty, vegetated, burning, or burnt.
    \item A vegetated cell will become burning if one of it's neighbours is
          burning.
    \item A burning cell will become burnt.
\end{enumerate}
where "neighbour" is defined as being in the 8-cell Moore neighbourhood of a
cell.

% section ff_impl (end)

\section{Experimentation} 
\label{sec:ff_exp}

Lorem ipsum dolor sit amet, consetetur sadipscing elitr, sed diam nonumy eirmod
tempor invidunt ut labore et dolore magna aliquyam erat, sed diam voluptua. At
vero eos et accusam et justo duo dolores et ea rebum. Stet clita kasd gubergren,
no sea takimata sanctus est Lorem ipsum dolor sit amet.

% section ff_exp (end)

% chapter ff (end)

\chapter{A grid-simulation of the spread of malaria}
\label{cha:mlaria}


In this grid-simulation a representation is made of how malaria can spread
through populations, and what the conditions are for malaria to die out or to
be succesful on a lot of iteration


To model such a complicated problem a few simplifying assumptions were made:

\begin{itemize}
    \item Every time iteration a mosquito moves to a neighboring cell when
        there is an vacant cell (the possible directions are up, down, left and
        right)
    \item Humans are stoic.
    \item Only one mosquito can occupy a cell at a time iteration, but both a
        human and a mosquito can be in the same cell
    \item Both humans and mosquitoes can die and for every died person or
        mosquito a new uninfected is born
    \item Humans will spawn next to already existing humans, and the same for
        mosquitoes
\end{itemize}

The chance of a human becoming infected or immune, or to become susceptible or
die is all determined by some rates.
The rates ar eused as follows:
o

For the following experiments a 30 X 30 grid was used in which experimentations
were made with the human density and mosquito density and the amount of
susceptible and unsusceptible humans that were started with


% chapter mlaria (end)

\end{document}
