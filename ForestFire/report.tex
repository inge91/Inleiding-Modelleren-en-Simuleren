\documentclass[a4paper]{report}

\usepackage[english]{babel}
\usepackage{amsmath}

\author{Maarten de Jonge \\
        Inge Becht}
\date{\today}
\title{Various Models and Simulations}

\begin{document}
\maketitle

\chapter{Simulating a forest fire with a cellular automaton}
\label{cha:ff}

\section{Implementation} 
\label{sec:ff_impl}

The simulation is implemented as a cellular automaton using the following simple
rules:
\begin{enumerate}
    \item A cell can be empty, vegetated, burning, or burnt.
    \item A vegetated cell will become burning if one of it's neighbours is
          burning.
    \item A burning cell will become burnt.
\end{enumerate}
where "neighbour" is defined as being in the 8-cell Moore neighbourhood of a
cell.

% section ff_impl (end)

\section{Experimentation} 
\label{sec:ff_exp}

Lorem ipsum dolor sit amet, consetetur sadipscing elitr, sed diam nonumy eirmod
tempor invidunt ut labore et dolore magna aliquyam erat, sed diam voluptua. At
vero eos et accusam et justo duo dolores et ea rebum. Stet clita kasd gubergren,
no sea takimata sanctus est Lorem ipsum dolor sit amet.

% section ff_exp (end)

% chapter ff (end)

\chapter{A grid-simulation of Africa}
\label{cha:mlaria}

Lorem ipsum dolor sit amet, consetetur sadipscing elitr, sed diam nonumy eirmod
tempor invidunt ut labore et dolore magna aliquyam erat, sed diam voluptua. At
vero eos et accusam et justo duo dolores et ea rebum. Stet clita kasd gubergren,
no sea takimata sanctus est Lorem ipsum dolor sit amet.

% chapter mlaria (end)

\end{document}
